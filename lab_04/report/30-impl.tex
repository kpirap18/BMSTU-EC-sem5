\chapter*{Контрольные вопросы}
\addcontentsline{toc}{chapter}{Контрольные вопросы}

\subsubsection{1. Назовите преимущества и недостатки XDMA и QDMA платформ.}

Преимущества QDMA:
\begin{itemize}
	\item позволяет передавать поток данных непосредственно в логику FPGA параллельно с их обработкой.
	\item предоставляет разработчикам прямое потоковое соединение с низкой задержкой между хостом и ядрами.
	\item включает высокопроизводительный DMA, который использует несколько очередей, оптимизированных как для передачи данных с высокой пропускной способностью, так и для передачи данных с большим количеством пакетов.
\end{itemize}

Недостатки XDMA:
\begin{itemize}
	\item требует, чтобы данные сначала были полностью перемещены из памяти хоста в память FPGA (DDRx4 DIMM или PLRAM), прежде чем логика FPGA сможет начать обработку данных, что влияет на задержку на запуска задачи.
\end{itemize}

\subsubsection{2. Назовите последовательность действий, необходимых для инициализации ускорителя со стороны хост-системы.}

\begin{enumerate}
	\item Хост получает все платформы.
	\item Хост выбирает имя платформы Xilinx.
	\item Хост получает Id устройства.
	\item Хост получает информацию об устройстве.
	\item Создается контекст для переменных.
	\item Создается команда для устройста-ускорителя.
\end{enumerate}

\subsubsection{3. Какова процедура запуска задания на исполнения в ускорительном ядре VINC.}

\begin{enumerate}
	\item Данные из .xclbin копируются из ОЗУ в локальную память ускорителя посредством DMA.
	\item В памяти устройства-ускорителя создается исполняемый файл.
	\item Те данные, которые подлежат обработке, копируются из ОЗУ в локальную память усокрителя посредством DMA.
	\item Указываются необходимые параметры и запускается программа на ускорителе.
	\item В конце выполняется чтение готовых данных. 
\end{enumerate}

\subsubsection{4. Опишите процесс линковки на основании содержимого файла v++\_*.log.}

\begin{enumerate}
	\item Анализ профиля устройства. Анализ конфигурационного файла. Поиск необходимых интерфейсов.
	\item FPGA linking synthesized kernels to platform
	\item FPGA logic optimization (оптимизация логики ПЛИС) для минимизации задержки.
	\item FPGA logic placement (размещение логики ПЛИС, то есть выбор конкретного мета для определенного логического блока). 
	\item FPGA routing (маршрутизация ПЛИС)
	\item FPGA bitstream generation (генерация битового потока ПЛИС, то есть генерация файла  [*.xclbin]).
\end{enumerate}