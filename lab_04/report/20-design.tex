\chapter*{Копии экранов моделирования исходного проекта VINC (исходная программа)}
\addcontentsline{toc}{chapter}{Копии экранов моделирования исходного проекта VINC(исходная программа)}

По умолчанию, в диаграмму (которую необходимо получить в приложении Vivado) добавлены сигналы шины AXI4 MM, представляющие собой 5 независимых каналов передачи сообщений, которые представлены в таблице \ref{tab:t1}.

\begin{table}[h]
	\begin{center}
		\caption{\label{tab:t1}Результаты замеров времени.}
		\begin{tabular}{|c|c|}
			\hline
			Канал передачи & Группы сигналов \\
			\hline
			Канал чтения адреса от ведущего к ведомому & m00\_axi\_ar* \\
			\hline
			Канал чтения данных от ведомого к ведущему & m00\_axi\_r* \\
			\hline
			Канал записи адреса записи от ведущего к ведомому & m00\_axi\_aw* \\
			\hline
			Канал запись данных от ведущего к ведомому & m00\_axi\_w* \\
			\hline
			Канал записи ответа от ведомого к ведущему & m00\_axi\_b* \\
			\hline
		\end{tabular}
	\end{center}
\end{table}

Каналы позволяют сформировать конвейерные транзакции чтения и записи. Последовательность событий транзакции чтения можно представить следующим образом: ARVALID→ ARREADY→ RVALID→ RREADY.

Последовательность событий транзакции записи: AWVALID→ AWREADY → WVALID → WREADY → BVALID → BREADY.

На рисунке \ref{img:i1} приведена транзакция чтения данных вектора на шине AXI4 MM из DDR памяти.

\img{50mm}{i1}{Транзакция чтения данных вектора на шине AXI4 MM из DDR памяти}
\newpage

На рисунке \ref{img:i2} приведена транзакция записи результата инкремента данных на шине AXI4 MM.

\img{90mm}{i2}{Транзакция записи результата инкремента данных на шине AXI4 MM}


На рисунке \ref{img:i3} приведен фрагмент кода модуля rtl\_kernel\_wizard\_0\_example\_adder.v с выполнением инкрементирования данных.

\img{20mm}{i3}{Код модуля rtl\_kernel\_wizard\_0\_example\_adder.v с выполнением инкрементирования данных}
\newpage

\chapter*{Копии экранов моделирования исходного проекта VINC (измененная программа)}
\addcontentsline{toc}{chapter}{Копии экранов моделирования исходного проекта VINC(измененная программа)}

В соответствии с вариантом 9 необходимо было изменить код проекта. Реализовать функцию \ref{eq:D}


\begin{equation}
	\label{eq:D}
	R[i] = min(A[i] - 4, 5)
\end{equation}

На рисунке \ref{img:i6} приведен код измененнной программы.

\img{30mm}{i6}{Измененный код модуля rtl\_kernel\_wizard\_0\_example\_adder.v}

Константы, которые используются в данном коде, представлены на рисунке \ref{img:i7}.

\img{45mm}{i7}{Константы}

На рисунке \ref{img:i4} приведена транзакция чтения данных вектора на шине AXI4 MM из DDR памяти.

\img{60mm}{i4}{Транзакция чтения данных вектора на шине AXI4 MM из DDR памяти}
\newpage

На рисунке \ref{img:i5} приведена транзакция записи результата инкремента данных на шине AXI4 MM.

\img{90mm}{i5}{Транзакция записи результата инкремента данных на шине AXI4 MM}


\chapter*{Сборка проекта}
\addcontentsline{toc}{chapter}{Сборка проекта}

Для сборки проекта необходимо было написать конфигурационный файл.

В конфигурационом файле указывается основная информация для работы компилятора v++:

\begin{enumerate}
	\item Количество и условные имена экземпляров ядер.
	\item Тактовая частота работы ядра.
	\item Для каждого ядра: выбор области SLR (SLR[0..2]), выбор DDR (DDR[0..3]) памяти, выбор высокопроизводительной памяти PLRAM( PLRAM[0,1,2]).
	\item Параметры синтеза и оптимизации проекта.
\end{enumerate}

На рисунке \ref{img:i8} представлен конфигурационный файл для сборки проекта.

\img{60mm}{i8}{Конфигурационный файл}

Содержимое файлов v++*.log и *.xclbin.info. приведено в приложениях.

\chapter*{Тестирование}
\addcontentsline{toc}{chapter}{Тестирование}

Для того, чтобы запустить тесты, необходимо изменить условие проверки в автоматически созданном программном модуле host\_example.cpp.

Часть кода модуля host\_example.cpp приведена на рисунке \ref{img:i9}. Было изменено условие проверки, на проверку, соответствующую моему варианту.

\img{60mm}{i9}{host\_example.cpp}

Тесты запускались с помощью утилиты xgdb. Результаты тестирования приведены на рисунке \ref{img:i10}.

\img{90mm}{i10}{Тестирование}

По результатам можно увидеть, что все тесты выполнены успешно.