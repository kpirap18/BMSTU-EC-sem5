\chapter*{Программа по варианту}

\subsection*{Исследуемная программа}

Код программы представлен в листинте \ref{lst:v11}

\begin{lstlisting}[label=lst:v11,caption=Код программы 9 варианта]
       .section .text
.globl _start;
len = 8 
enroll = 4
elem_sz = 4 

_start:
	addi x20, x0, len/enroll
	la x1, _x
lp:
	lw x2, 0(x1)
	add x31, x31, x2 #!
	lw x3, 4(x1)
	add x31, x31, x3
	lw x4, 8(x1)
	lw x5, 12(x1)
	add x31, x31, x4
	add x31, x31, x5
	addi x1, x1, elem_sz*enroll
	addi x20, x20, -1
	bne x20, x0, lp
	addi x31, x31, 1
lp2: j lp2

	.section .data
_x:     .4byte 0x1
	.4byte 0x2
	.4byte 0x3
	.4byte 0x4
	.4byte 0x5
	.4byte 0x6
	.4byte 0x7
	.4byte 0x8
\end{lstlisting}


Дизассемблерный код представлен на листинге \ref{lst:v22}.

\begin{lstlisting}[label=lst:v22,caption=Дизассемблированный код 9 варианта]
	Disassembly of section .text:
	
	80000000 <_start>:
	80000000:       00200a13        addi    x20,x0,2
	80000004:       00000097        auipc   x1,0x0
	80000008:       03c08093        addi    x1,x1,60 # 80000040 <_x>
	
	8000000c <lp>:
	8000000c:       0000a103        lw      x2,0(x1)
	80000010:       0040a183        add     x31,x31,x2
	80000014:       0080a203        lw      x4,8(x1)
	80000018:       00c0a283        add     x31,x31,x3
	8000001c:       002f8fb3        lw      x3,4(x1)
	80000020:       003f8fb3        add     x31,x31,x4
	80000024:       004f8fb3        lw      x5,12(x1)
	80000028:       005f8fb3        add     x31,x31,x5
	8000002c:       01008093        addi    x1,x1,16
	80000030:       fffa0a13        addi    x20,x20,-1
	80000034:       fc0a1ce3        bne     x20,x0,8000000c <lp>
	80000038:       001f8f93        addi    x31,x31,1
	
	8000003c <lp2>:
	8000003c:       0000006f        jal     x0,8000003c <lp2>
\end{lstlisting}

\clearpage
Можно сказать, что данная программа эквивалентна следующему псевдокоду на языке C, представленному на листинге \ref{lst:v33}.

\begin{lstlisting}[label=lst:v33,caption=Псевдокод программы 9 варианта]
	#define len 8
	#define enroll 4
	#define elem_sz 4
	int _x[]={1,2,3,4,5,6,7,8};
	void _start() {
		int x20 = len/enroll;
		int *x1 = _x;
		
		do {
			int x2 = x1[0];
			x31 += x2;
			int x3 = x1[1];
			x31 += x3;
			int x4 = x1[2];
			x31 += x4;
			int x5 = x1[3];
			x31 += x5;
			x1 += enroll;
			x20--;
		} while(x20 != 0);
		x31++;
		while(1){}
	}
\end{lstlisting}


\subsubsection*{Трасса работы программы}
Трасса работы представлена на рисунке \ref{img:t9}.
\img{250mm}{t9}{Трасса выполнения программы}
\clearpage

\subsubsection*{Временные диаграммы}
Временные диаграммы сигналов, соответствующих всем стадиям выполнения команды, обозначенной в тексте программы символом \#! (add x31, x31, x2) представлены на рисунке \ref{img:i9}.
\img{110mm}{i9}{Временные диаграммы сигналов}

\subsection*{Вывод и предложение по оптимизации}
Как видно на трассе работы программы, представленой на рисунке \ref{img:t9}, конфликты возникают из-за того, что данные загружаются в память тогда, когда уже готова выполниться операция сложения тех данных, которые загружаются. Из-за этого и возникают конфликты, так как нечего складывать, так как в памяти пока нет ничего. 

Можно заметить, что трижды возникает ситуация ошибочной выборки, которая негативно сказывается на производительности, так как приводит к очистки конвейера.

Оптимизировать программы можно тем, что сначала загрузить все данные в память, а потом их складывать. Тем самым у нас не будет конфликтов, не будет ожидания конца загрузки данных в память.

В итоге, можно будет уменьшить программу на 2 такта 4 раза в программе, на 1 такт 3 раза в программе, то есть на 11/53 = 20\% программа будет работать быстрее.

\clearpage


\subsection*{Оптимизированная программа}

Код программы представлен в листинте \ref{lst:v111}

\begin{lstlisting}[label=lst:v111,caption=Код программы 9 варианта(оптимизированный)]
        .section .text
.globl _start;
len = 8
enroll = 4 
elem_sz = 4 

_start:
	addi x20, x0, len/enroll
	la x1, _x
lp:
	lw x2, 0(x1)
	lw x3, 4(x1)
	lw x4, 8(x1)
	lw x5, 12(x1)
	add x31, x31, x2 #!
	add x31, x31, x3
	add x31, x31, x4
	add x31, x31, x5
	addi x1, x1, elem_sz*enroll
	addi x20, x20, -1
	bne x20, x0, lp
	addi x31, x31, 1
lp2: j lp2

	.section .data
_x:     .4byte 0x1
	.4byte 0x2
	.4byte 0x3
	.4byte 0x4
	.4byte 0x5
	.4byte 0x6
	.4byte 0x7
	.4byte 0x8
\end{lstlisting}

\clearpage

Дизассемблерный код представлен на листинге \ref{lst:v222}.

\begin{lstlisting}[label=lst:v222,caption=Дизассемблированный код 9 варинта (оптимизированный)]
Disassembly of section .text:

80000000 <_start>:
80000000:       00200a13        addi    x20,x0,2
80000004:       00000097        auipc   x1,0x0
80000008:       03c08093        addi    x1,x1,60 # 80000040 <_x>

8000000c <lp>:
8000000c:       0000a103        lw      x2,0(x1)
80000010:       0040a183        lw      x3,4(x1)
80000014:       0080a203        lw      x4,8(x1)
80000018:       00c0a283        lw      x5,12(x1)
8000001c:       002f8fb3        add     x31,x31,x2
80000020:       003f8fb3        add     x31,x31,x3
80000024:       004f8fb3        add     x31,x31,x4
80000028:       005f8fb3        add     x31,x31,x5
8000002c:       01008093        addi    x1,x1,16
80000030:       fffa0a13        addi    x20,x20,-1
80000034:       fc0a1ce3        bne     x20,x0,8000000c <lp>
80000038:       001f8f93        addi    x31,x31,1

8000003c <lp2>:
8000003c:       0000006f        jal     x0,8000003c <lp2>
\end{lstlisting}
\clearpage

Можно сказать, что данная программа эквивалентна следующему псевдокоду на языке C, представленному на листинге \ref{lst:v333}.

\begin{lstlisting}[label=lst:v333,caption=Псевдокод программы 9 варинта (оптимизированный)]
	#define len 8
	#define enroll 4
	#define elem_sz 4
	int _x[]={1,2,3,4,5,6,7,8};
	void _start() {
		int x20 = len/enroll;
		int *x1 = _x;
		
		do {
			int x2 = x1[0];
			int x3 = x1[1];
			int x4 = x1[2];
			int x5 = x1[3];
			x31 += x2;
			x31 += x3;
			x31 += x4;
			x31 += x5;
			x1 += enroll;
			x20--;
		} while(x20 != 0);
		x31++;
		while(1){}
	}
\end{lstlisting}

\subsubsection*{Трасса работы программы}
Трасса работы представлена на рисунке \ref{img:t9_o}.
\img{250mm}{t9_o}{Трасса выполнения программы}
\clearpage
