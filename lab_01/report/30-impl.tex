\chapter*{Практическая часть}

\section*{Функциональная схема разрабатываемой системы на кристале}

Функциональная схема расзарабатываемой системы на кристале представлена на рисунке \ref{img:p}.

\img{90mm}{p}{Функциональная схема разрабатываемой системы на
	кристалле}

Система на кристалле состоит из следующих блоков.

\begin{enumerate}
	\item Микропроцессорное ядро Nios II/e выполняет функции управления
	системой.

	\item Внутренняя оперативная память СНК, используемая для хранения программы
	управления и данных. 
	\item Системная шина Avalon обеспечивает связность всех компонентов системы.
	\item Блок синхронизации и сброса обеспечивает обработку входных сигналов сброса и
	синхронизации и распределение их в системе. Внутренний сигнал сброса
	синхронизирован и имеет необходимую для системы длительность.
	\item Блок идентификации версии проекта обеспечивает хранение и выдачу уникального идентификатора версии, который используется программой управления при инициализации системы.
	\item Контроллер UART обеспечивает прием и передачу информации по
	интерфейсу RS232.
\end{enumerate}
\clearpage

\section*{Создание нового модуля системы на кристале QSYS}


\begin{enumerate}
	\item Был создан новый модуль Qsys.	
	\item Установлена частота внешнего сигнала синхронизации 50 000 000 Гц.
	\item Добавлен в проект модуль синхронизируемого микропроцессорного ядра Nios2.
	\item Добавлен в проект модуль ОЗУ программ и данных.
	\item Добавлены компоненты Avalon System ID, Avalon UART.
	\item Создана сеть синхронизации и сбоса системы.
	\item Сигналы TX и RX экспортированы во внешние порты.
    
	\item Назначены базовые адреса устройств.
\end{enumerate}

Итог выполненных действий показан на рисунке \ref{img:p1}.

\img{90mm}{p1}{Модуль Qsys}
\clearpage

\section*{Назначение портами проекта контакты микросхемы} 

Назначены котакты в соответствии с таблицей из методических указаний. 

Таблица представлена на рисунке \ref{img:pp}


\img{50mm}{pp}{Таблица из методических указаний}


Был выполнен синтез проекта.

Результат представлен на рисунке \ref{img:p2}.

\img{100mm}{p2}{Модуль Pin Planner}
\clearpage

\section*{Сoздание проекта Nios2}


В файле  $hello\_world\_small.c$ добавлен код эхо-программы приема-передачи по интерфейсу RS232.

Также был создан образ ОС HAL с драйверами устройств, используемых в аппаратном проекте.

Результат представлен на рисунке \ref{img:p3}.

\img{100mm}{p3}{Проект Nios2}
\clearpage

\section*{Подключение к ПК отладочной платы с ПЛИС EPC2C20 и вывод необходимого сообщения на экран}
К ПК была подключена отладочная плата с ПЛИС EPC2C20.


Была выполнена верификация проекта с использованием программы терминала. Доработан код проекта с использованием необходимых библиотек (представлены ниже).

 $\#include "system.h"$
 
 $\#include "altera\_avalon\_sysid\_qsys.h"$
 
 $\#include "altera\_avalon\_sysid\_qsys\_regs.h"$
 
 
Доработанный код проекта, а также вывод сообщения с номером группы (52) представлены на рисунке \ref{img:p4}.

\img{100mm}{p4}{Код программной части проекта}
\clearpage