\chapter*{Контрольные вопросы}
\addcontentsline{toc}{chapter}{Контрольные вопросы}

\subsubsection{1. Назовите преимущества и недостатки аппаратных ускорителей на ПЛИС по сравнению с CPU и графическими ускорителями?}
Достоинствами данной системы являются:
\begin{itemize}
	\item низкая стоимость в сравнению с аппаратными ускорителями;
	\item большая частота эмуляции;
	\item компактность.
\end{itemize}

Основными недостатками аппаратных эмуляторов на ПЛИС являются: 
\begin{itemize}
	\item необходимость перекомпиляции проекта и переконфигурации ПЛИС при любом исправлении содержимого проекта;
	\item наличие специализированного программного обеспечения для разделения модели микросхемы на части для загрузки в отдельные ПЛИС.
\end{itemize}


\subsubsection{2. Назовите основные способы оптимизации циклических конструкций ЯВУ, реализуемых в виде аппаратных ускорителей?}


\subsubsection{3. Назовите этапы работы программной части ускорителя в хост системе?}


\subsubsection{4. В чем заключается процесс отладки для вариантов сборки Emulation-SW, Emulation-HW и Hardware?}


\subsubsection{5. Какие инструменты и средства анализа результатов синтеза возможно использовать в Vitis HLS для оптимизации ускорителей?}
